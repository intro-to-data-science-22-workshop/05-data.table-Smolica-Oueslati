% Options for packages loaded elsewhere
\PassOptionsToPackage{unicode}{hyperref}
\PassOptionsToPackage{hyphens}{url}
%
\documentclass[
]{article}
\usepackage{amsmath,amssymb}
\usepackage{lmodern}
\usepackage{iftex}
\ifPDFTeX
  \usepackage[T1]{fontenc}
  \usepackage[utf8]{inputenc}
  \usepackage{textcomp} % provide euro and other symbols
\else % if luatex or xetex
  \usepackage{unicode-math}
  \defaultfontfeatures{Scale=MatchLowercase}
  \defaultfontfeatures[\rmfamily]{Ligatures=TeX,Scale=1}
\fi
% Use upquote if available, for straight quotes in verbatim environments
\IfFileExists{upquote.sty}{\usepackage{upquote}}{}
\IfFileExists{microtype.sty}{% use microtype if available
  \usepackage[]{microtype}
  \UseMicrotypeSet[protrusion]{basicmath} % disable protrusion for tt fonts
}{}
\makeatletter
\@ifundefined{KOMAClassName}{% if non-KOMA class
  \IfFileExists{parskip.sty}{%
    \usepackage{parskip}
  }{% else
    \setlength{\parindent}{0pt}
    \setlength{\parskip}{6pt plus 2pt minus 1pt}}
}{% if KOMA class
  \KOMAoptions{parskip=half}}
\makeatother
\usepackage{xcolor}
\usepackage[margin=1in]{geometry}
\usepackage{color}
\usepackage{fancyvrb}
\newcommand{\VerbBar}{|}
\newcommand{\VERB}{\Verb[commandchars=\\\{\}]}
\DefineVerbatimEnvironment{Highlighting}{Verbatim}{commandchars=\\\{\}}
% Add ',fontsize=\small' for more characters per line
\usepackage{framed}
\definecolor{shadecolor}{RGB}{248,248,248}
\newenvironment{Shaded}{\begin{snugshade}}{\end{snugshade}}
\newcommand{\AlertTok}[1]{\textcolor[rgb]{0.94,0.16,0.16}{#1}}
\newcommand{\AnnotationTok}[1]{\textcolor[rgb]{0.56,0.35,0.01}{\textbf{\textit{#1}}}}
\newcommand{\AttributeTok}[1]{\textcolor[rgb]{0.77,0.63,0.00}{#1}}
\newcommand{\BaseNTok}[1]{\textcolor[rgb]{0.00,0.00,0.81}{#1}}
\newcommand{\BuiltInTok}[1]{#1}
\newcommand{\CharTok}[1]{\textcolor[rgb]{0.31,0.60,0.02}{#1}}
\newcommand{\CommentTok}[1]{\textcolor[rgb]{0.56,0.35,0.01}{\textit{#1}}}
\newcommand{\CommentVarTok}[1]{\textcolor[rgb]{0.56,0.35,0.01}{\textbf{\textit{#1}}}}
\newcommand{\ConstantTok}[1]{\textcolor[rgb]{0.00,0.00,0.00}{#1}}
\newcommand{\ControlFlowTok}[1]{\textcolor[rgb]{0.13,0.29,0.53}{\textbf{#1}}}
\newcommand{\DataTypeTok}[1]{\textcolor[rgb]{0.13,0.29,0.53}{#1}}
\newcommand{\DecValTok}[1]{\textcolor[rgb]{0.00,0.00,0.81}{#1}}
\newcommand{\DocumentationTok}[1]{\textcolor[rgb]{0.56,0.35,0.01}{\textbf{\textit{#1}}}}
\newcommand{\ErrorTok}[1]{\textcolor[rgb]{0.64,0.00,0.00}{\textbf{#1}}}
\newcommand{\ExtensionTok}[1]{#1}
\newcommand{\FloatTok}[1]{\textcolor[rgb]{0.00,0.00,0.81}{#1}}
\newcommand{\FunctionTok}[1]{\textcolor[rgb]{0.00,0.00,0.00}{#1}}
\newcommand{\ImportTok}[1]{#1}
\newcommand{\InformationTok}[1]{\textcolor[rgb]{0.56,0.35,0.01}{\textbf{\textit{#1}}}}
\newcommand{\KeywordTok}[1]{\textcolor[rgb]{0.13,0.29,0.53}{\textbf{#1}}}
\newcommand{\NormalTok}[1]{#1}
\newcommand{\OperatorTok}[1]{\textcolor[rgb]{0.81,0.36,0.00}{\textbf{#1}}}
\newcommand{\OtherTok}[1]{\textcolor[rgb]{0.56,0.35,0.01}{#1}}
\newcommand{\PreprocessorTok}[1]{\textcolor[rgb]{0.56,0.35,0.01}{\textit{#1}}}
\newcommand{\RegionMarkerTok}[1]{#1}
\newcommand{\SpecialCharTok}[1]{\textcolor[rgb]{0.00,0.00,0.00}{#1}}
\newcommand{\SpecialStringTok}[1]{\textcolor[rgb]{0.31,0.60,0.02}{#1}}
\newcommand{\StringTok}[1]{\textcolor[rgb]{0.31,0.60,0.02}{#1}}
\newcommand{\VariableTok}[1]{\textcolor[rgb]{0.00,0.00,0.00}{#1}}
\newcommand{\VerbatimStringTok}[1]{\textcolor[rgb]{0.31,0.60,0.02}{#1}}
\newcommand{\WarningTok}[1]{\textcolor[rgb]{0.56,0.35,0.01}{\textbf{\textit{#1}}}}
\usepackage{graphicx}
\makeatletter
\def\maxwidth{\ifdim\Gin@nat@width>\linewidth\linewidth\else\Gin@nat@width\fi}
\def\maxheight{\ifdim\Gin@nat@height>\textheight\textheight\else\Gin@nat@height\fi}
\makeatother
% Scale images if necessary, so that they will not overflow the page
% margins by default, and it is still possible to overwrite the defaults
% using explicit options in \includegraphics[width, height, ...]{}
\setkeys{Gin}{width=\maxwidth,height=\maxheight,keepaspectratio}
% Set default figure placement to htbp
\makeatletter
\def\fps@figure{htbp}
\makeatother
\setlength{\emergencystretch}{3em} % prevent overfull lines
\providecommand{\tightlist}{%
  \setlength{\itemsep}{0pt}\setlength{\parskip}{0pt}}
\setcounter{secnumdepth}{-\maxdimen} % remove section numbering
\ifLuaTeX
  \usepackage{selnolig}  % disable illegal ligatures
\fi
\IfFileExists{bookmark.sty}{\usepackage{bookmark}}{\usepackage{hyperref}}
\IfFileExists{xurl.sty}{\usepackage{xurl}}{} % add URL line breaks if available
\urlstyle{same} % disable monospaced font for URLs
\hypersetup{
  pdftitle={Data wrangling at scale with data.table},
  pdfauthor={Gresa Smolica \& Amin Oueslati},
  hidelinks,
  pdfcreator={LaTeX via pandoc}}

\title{Data wrangling at scale with data.table}
\author{Gresa Smolica \& Amin Oueslati}
\date{2022-11-16}

\begin{document}
\maketitle

{
\setcounter{tocdepth}{3}
\tableofcontents
}
\begin{center}\rule{0.5\linewidth}{0.5pt}\end{center}

\hypertarget{putting-data.table-to-the-speed-test}{%
\section{Putting data.table to the speed
test}\label{putting-data.table-to-the-speed-test}}

Let's start by installing/loading all the packages we need.

\begin{Shaded}
\begin{Highlighting}[]
\NormalTok{pacman}\SpecialCharTok{::}\FunctionTok{p\_load}\NormalTok{(data.table, learningtower, tidyverse)}
\end{Highlighting}
\end{Shaded}

Before we start, let's go back to our claims regarding data.table's
speed? You don't believe us? See for yourself!

The examplary data on data.table's speec is extracted from ``EDAV Fall
2021 Tues/Thurs Community Contributions:

\begin{Shaded}
\begin{Highlighting}[]
\DocumentationTok{\#\# read a csv into a data.frame}
\NormalTok{start }\OtherTok{\textless{}{-}} \FunctionTok{Sys.time}\NormalTok{()}
\NormalTok{df }\OtherTok{\textless{}{-}} \FunctionTok{read.csv}\NormalTok{(}\StringTok{"https://raw.githubusercontent.com/CSSEGISandData/COVID{-}19/master/csse\_covid\_19\_data/csse\_covid\_19\_time\_series/time\_series\_covid19\_confirmed\_US.csv"}\NormalTok{)}
\NormalTok{end }\OtherTok{\textless{}{-}} \FunctionTok{Sys.time}\NormalTok{()}
\FunctionTok{print}\NormalTok{(end }\SpecialCharTok{{-}}\NormalTok{ start)}
\end{Highlighting}
\end{Shaded}

\begin{verbatim}
## Time difference of 11.84003 secs
\end{verbatim}

\begin{Shaded}
\begin{Highlighting}[]
\DocumentationTok{\#\# reading a csv into a tibble}
\NormalTok{start }\OtherTok{\textless{}{-}} \FunctionTok{Sys.time}\NormalTok{()}
\NormalTok{tb }\OtherTok{\textless{}{-}} \FunctionTok{read\_csv}\NormalTok{(}\StringTok{"https://raw.githubusercontent.com/CSSEGISandData/COVID{-}19/master/csse\_covid\_19\_data/csse\_covid\_19\_time\_series/time\_series\_covid19\_confirmed\_US.csv"}\NormalTok{)}
\end{Highlighting}
\end{Shaded}

\begin{verbatim}
## Rows: 3342 Columns: 1040
## -- Column specification --------------------------------------------------------
## Delimiter: ","
## chr    (6): iso2, iso3, Admin2, Province_State, Country_Region, Combined_Key
## dbl (1034): UID, code3, FIPS, Lat, Long_, 1/22/20, 1/23/20, 1/24/20, 1/25/20...
## 
## i Use `spec()` to retrieve the full column specification for this data.
## i Specify the column types or set `show_col_types = FALSE` to quiet this message.
\end{verbatim}

\begin{Shaded}
\begin{Highlighting}[]
\NormalTok{end }\OtherTok{\textless{}{-}} \FunctionTok{Sys.time}\NormalTok{()}
\FunctionTok{print}\NormalTok{(end }\SpecialCharTok{{-}}\NormalTok{ start)}
\end{Highlighting}
\end{Shaded}

\begin{verbatim}
## Time difference of 1.531926 secs
\end{verbatim}

\begin{Shaded}
\begin{Highlighting}[]
\CommentTok{\# reading a csv into a data.table}
\NormalTok{start }\OtherTok{\textless{}{-}} \FunctionTok{Sys.time}\NormalTok{()}
\NormalTok{dt }\OtherTok{\textless{}{-}} \FunctionTok{fread}\NormalTok{(}\StringTok{"https://raw.githubusercontent.com/CSSEGISandData/COVID{-}19/master/csse\_covid\_19\_data/csse\_covid\_19\_time\_series/time\_series\_covid19\_confirmed\_US.csv"}\NormalTok{)}
\NormalTok{end }\OtherTok{\textless{}{-}} \FunctionTok{Sys.time}\NormalTok{()}
\FunctionTok{print}\NormalTok{(end }\SpecialCharTok{{-}}\NormalTok{ start)}
\end{Highlighting}
\end{Shaded}

\begin{verbatim}
## Time difference of 0.6541331 secs
\end{verbatim}

\hypertarget{data-wrangling-the-basics}{%
\section{Data wrangling: The Basics}\label{data-wrangling-the-basics}}

We use data from the learningtower package, which provides easy access
to a subset of variables from the PISA educational study for the years
2000-2008. The data collection is orchestrated by the OECD. In this
workshop we will use both the student and school datasets. While the
former contains data at the student-level, the latter has schools as its
units of observation.

\begin{Shaded}
\begin{Highlighting}[]
\CommentTok{\# load the student data for 2018}
\FunctionTok{load\_student}\NormalTok{(}\StringTok{"2018"}\NormalTok{) }
\end{Highlighting}
\end{Shaded}

\begin{verbatim}
## Downloading year 2018...
\end{verbatim}

\begin{verbatim}
## # A tibble: 612,004 x 22
##    year  country school_id studen~1 mothe~2 fathe~3 gender compu~4 inter~5  math
##    <fct> <fct>   <fct>     <fct>    <fct>   <fct>   <fct>  <fct>   <fct>   <dbl>
##  1 2018  ALB     800002    800251   ISCED ~ ISCED ~ male   yes     yes      490.
##  2 2018  ALB     800002    800402   ISCED 2 ISCED 2 male   yes     yes      462.
##  3 2018  ALB     800002    801902   ISCED 2 ISCED 2 female no      no       407.
##  4 2018  ALB     800002    803546   ISCED 2 ISCED 2 male   no      no       483.
##  5 2018  ALB     800002    804776   ISCED 2 ISCED ~ male   yes     yes      460.
##  6 2018  ALB     800002    804825   ISCED 2 ISCED 2 female yes     yes      367.
##  7 2018  ALB     800002    804983   <NA>    <NA>    female <NA>    <NA>     411.
##  8 2018  ALB     800002    805287   ISCED 2 ISCED 2 male   no      no       441.
##  9 2018  ALB     800002    805601   ISCED 2 ISCED 2 female yes     yes      506.
## 10 2018  ALB     800002    806295   <NA>    <NA>    female <NA>    <NA>     412.
## # ... with 611,994 more rows, 12 more variables: read <dbl>, science <dbl>,
## #   stu_wgt <dbl>, desk <fct>, room <fct>, dishwasher <fct>, television <fct>,
## #   computer_n <fct>, car <fct>, book <fct>, wealth <dbl>, escs <dbl>, and
## #   abbreviated variable names 1: student_id, 2: mother_educ, 3: father_educ,
## #   4: computer, 5: internet
## # i Use `print(n = ...)` to see more rows, and `colnames()` to see all variable names
\end{verbatim}

\begin{Shaded}
\begin{Highlighting}[]
\CommentTok{\# load the school data for all years}
\FunctionTok{data}\NormalTok{(}\StringTok{"school"}\NormalTok{)}
\end{Highlighting}
\end{Shaded}

Now, let's convert our datasets into data.table objects and assign them
to a variable to make our life easier:

\begin{Shaded}
\begin{Highlighting}[]
\CommentTok{\# assigning the datasets to objects}
\NormalTok{student\_data }\OtherTok{\textless{}{-}} \FunctionTok{as.data.table}\NormalTok{(}\FunctionTok{load\_student}\NormalTok{(}\StringTok{"2018"}\NormalTok{))}
\end{Highlighting}
\end{Shaded}

\begin{verbatim}
## Downloading year 2018...
\end{verbatim}

\begin{Shaded}
\begin{Highlighting}[]
\NormalTok{school\_data }\OtherTok{\textless{}{-}} \FunctionTok{as.data.table}\NormalTok{(school)}

\CommentTok{\# let\textquotesingle{}s inspect our datasets}
\FunctionTok{dim}\NormalTok{(student\_data) }\CommentTok{\# so our dataset has 612,004 rows/observations and 22 columns/variables. Awesome!}
\end{Highlighting}
\end{Shaded}

\begin{verbatim}
## [1] 612004     22
\end{verbatim}

\begin{Shaded}
\begin{Highlighting}[]
\FunctionTok{dim}\NormalTok{(school\_data) }\CommentTok{\# 109,756 rows and 13 columns. Great!}
\end{Highlighting}
\end{Shaded}

\begin{verbatim}
## [1] 109756     13
\end{verbatim}

Before we get into the wrangling, we should remind ourselves that
data.table works through the following syntax:

DT{[}i, j, by{]}: * i shows ``on which rows'' * j indicates ``what to
do'' * by means ``grouped by what?''

\hypertarget{sub-setting-rows}{%
\subsection{Sub-setting rows}\label{sub-setting-rows}}

{Tidyverse equivalent: filter}

We sub-set rows by defining the corresponding condition in i:

\begin{Shaded}
\begin{Highlighting}[]
\CommentTok{\# sub{-}set all male students from Albania}
\NormalTok{male\_albanian\_students }\OtherTok{\textless{}{-}}\NormalTok{ student\_data[country }\SpecialCharTok{==} \StringTok{"ALB"} \SpecialCharTok{\&}\NormalTok{ gender }\SpecialCharTok{==} \StringTok{"male"}\NormalTok{]}
\NormalTok{male\_albanian\_students [}\DecValTok{1}\SpecialCharTok{:}\DecValTok{5}\NormalTok{]}
\end{Highlighting}
\end{Shaded}

\begin{verbatim}
##    year country school_id student_id mother_educ father_educ gender computer
## 1: 2018     ALB    800002     800251    ISCED 3A    ISCED 3A   male      yes
## 2: 2018     ALB    800002     800402     ISCED 2     ISCED 2   male      yes
## 3: 2018     ALB    800002     803546     ISCED 2     ISCED 2   male       no
## 4: 2018     ALB    800002     804776     ISCED 2    ISCED 3A   male      yes
## 5: 2018     ALB    800002     805287     ISCED 2     ISCED 2   male       no
##    internet    math    read science  stu_wgt desk room dishwasher television
## 1:      yes 490.187 375.984 445.039 13.51452  yes  yes       <NA>         3+
## 2:      yes 462.464 434.352 421.731 13.51452  yes  yes       <NA>          1
## 3:       no 482.501 425.131 515.942 13.51452  yes   no       <NA>          0
## 4:      yes 459.804 306.028 328.261 13.51452  yes  yes       <NA>          2
## 5:       no 441.037 271.213 391.562 13.51452   no  yes       <NA>          1
##    computer_n  car  book  wealth    escs
## 1:          1    2  0-10 -0.0996  0.6747
## 2:          1    2 11-25 -0.7221 -0.7566
## 3:          0 <NA>  0-10 -7.0376 -3.1843
## 4:          1    0 11-25 -1.8375 -1.7557
## 5:          0    0  0-10 -3.3724 -3.2481
\end{verbatim}

As you can see, a comma after the condition in ``i''is not required. But
student\_data{[}country == ``ALB'' \& gender == ``male''\textbf{,}{]}
would work just fine.

\hypertarget{ordering-rows}{%
\subsection{Ordering rows}\label{ordering-rows}}

Exploring the power of (-) in data.table:

\begin{Shaded}
\begin{Highlighting}[]
\CommentTok{\# order students in increasing order by their math score}
\NormalTok{incr\_order }\OtherTok{\textless{}{-}}\NormalTok{ male\_albanian\_students[}\FunctionTok{order}\NormalTok{(math)]}
\end{Highlighting}
\end{Shaded}

\begin{Shaded}
\begin{Highlighting}[]
\CommentTok{\# order students in decreasing order by their math score}
\NormalTok{decr\_ord }\OtherTok{\textless{}{-}}\NormalTok{ male\_albanian\_students[}\FunctionTok{order}\NormalTok{(}\SpecialCharTok{{-}}\NormalTok{math)]}
\end{Highlighting}
\end{Shaded}

\hypertarget{between-and-inrange}{%
\subsection{\%between\% and \%inrange\%}\label{between-and-inrange}}

\%between\% and \%inrange\% are two special operators that are useful to
subset rows conditional on a value falling within a certain range. The
two operators accept both scalars and vectors. When used with a scalar,
\%between\% and \%inrange\% perform identically.

\begin{Shaded}
\begin{Highlighting}[]
\CommentTok{\# when used with a scalar, \%between\% and \%inrange\% perform identically}
\NormalTok{between\_scaler }\OtherTok{\textless{}{-}}\NormalTok{ male\_albanian\_students[math }\SpecialCharTok{\%between\%} \FunctionTok{c}\NormalTok{(}\DecValTok{500}\NormalTok{, }\DecValTok{600}\NormalTok{)]}
\NormalTok{inrange\_scaler }\OtherTok{\textless{}{-}}\NormalTok{ male\_albanian\_students[math }\SpecialCharTok{\%inrange\%} \FunctionTok{c}\NormalTok{(}\DecValTok{500}\NormalTok{, }\DecValTok{600}\NormalTok{)]}
\FunctionTok{length}\NormalTok{(between\_scaler) }\SpecialCharTok{==} \FunctionTok{length}\NormalTok{(inrange\_scaler)}
\end{Highlighting}
\end{Shaded}

\begin{verbatim}
## [1] TRUE
\end{verbatim}

However, both operators also accept vectors, in which case they operate
differently:

\%between\% evaluates each row and returns T/F if the value of interest
falls between the two vectors - the range defined by the two vectors
varies for each row.

\%inrange\% on the other hand takes the minimum from the lower-bound
(first vector) and the maximum from the upper-bound (second vector),
forms a static range and then evaluates whether the value in a given row
falls within this range.

\begin{Shaded}
\begin{Highlighting}[]
\NormalTok{between\_vector }\OtherTok{\textless{}{-}}\NormalTok{ male\_albanian\_students[math }\SpecialCharTok{\%between\%} \FunctionTok{list}\NormalTok{(science, read)]}
\NormalTok{between\_vector[}\DecValTok{1}\SpecialCharTok{:}\DecValTok{5}\NormalTok{] }\CommentTok{\# 188 rows}
\end{Highlighting}
\end{Shaded}

\begin{verbatim}
##    year country school_id student_id mother_educ father_educ gender computer
## 1: 2018     ALB    800019     801814     ISCED 2    ISCED 3A   male       no
## 2: 2018     ALB    800028     801470    ISCED 3A    ISCED 3A   male      yes
## 3: 2018     ALB    800029     800519    ISCED 3A ISCED 3B, C   male       no
## 4: 2018     ALB    800029     801021    ISCED 3A ISCED 3B, C   male       no
## 5: 2018     ALB    800030     803960     ISCED 2 ISCED 3B, C   male      yes
##    internet    math    read science stu_wgt desk room dishwasher television
## 1:       no 366.898 409.533 366.636 7.30754  yes  yes       <NA>          2
## 2:      yes 510.842 514.861 460.068 1.49627  yes  yes       <NA>          2
## 3:      yes 508.325 516.116 489.465 6.86134   no  yes       <NA>         3+
## 4:      yes 438.600 443.990 431.792 6.86134  yes  yes       <NA>          2
## 5:      yes 451.147 478.775 425.916 3.45567  yes  yes       <NA>         3+
##    computer_n  car   book  wealth    escs
## 1:          0 <NA>   0-10 -2.2853 -1.7831
## 2:          2 <NA> 26-100 -0.6582  0.1299
## 3:          0    0 26-100 -1.4801 -0.9225
## 4:          0    0 26-100 -2.1296 -1.5780
## 5:          1    1  11-25 -0.9061  0.4493
\end{verbatim}

\begin{Shaded}
\begin{Highlighting}[]
\NormalTok{inrange\_vector }\OtherTok{\textless{}{-}}\NormalTok{ male\_albanian\_students[math }\SpecialCharTok{\%inrange\%} \FunctionTok{list}\NormalTok{(science, read)]}
\NormalTok{inrange\_vector[}\DecValTok{1}\SpecialCharTok{:}\DecValTok{5}\NormalTok{] }\CommentTok{\# 3,182 rows, the math scores of all but one student fall inside the range}
\end{Highlighting}
\end{Shaded}

\begin{verbatim}
##    year country school_id student_id mother_educ father_educ gender computer
## 1: 2018     ALB    800002     800251    ISCED 3A    ISCED 3A   male      yes
## 2: 2018     ALB    800002     800402     ISCED 2     ISCED 2   male      yes
## 3: 2018     ALB    800002     803546     ISCED 2     ISCED 2   male       no
## 4: 2018     ALB    800002     804776     ISCED 2    ISCED 3A   male      yes
## 5: 2018     ALB    800002     805287     ISCED 2     ISCED 2   male       no
##    internet    math    read science  stu_wgt desk room dishwasher television
## 1:      yes 490.187 375.984 445.039 13.51452  yes  yes       <NA>         3+
## 2:      yes 462.464 434.352 421.731 13.51452  yes  yes       <NA>          1
## 3:       no 482.501 425.131 515.942 13.51452  yes   no       <NA>          0
## 4:      yes 459.804 306.028 328.261 13.51452  yes  yes       <NA>          2
## 5:       no 441.037 271.213 391.562 13.51452   no  yes       <NA>          1
##    computer_n  car  book  wealth    escs
## 1:          1    2  0-10 -0.0996  0.6747
## 2:          1    2 11-25 -0.7221 -0.7566
## 3:          0 <NA>  0-10 -7.0376 -3.1843
## 4:          1    0 11-25 -1.8375 -1.7557
## 5:          0    0  0-10 -3.3724 -3.2481
\end{verbatim}

\%between\% checks whether the math score of a given students falls
between his/her reading and science score, while \%inrange\% evaluates
whether he math score of a given student falls between the lowest
science and highest reading score of \textbf{all} students.

\hypertarget{extracting-columns}{%
\subsection{Extracting columns}\label{extracting-columns}}

{Tidyverse equivalent: select}

The logic is fairly straightforward. The main thing to keep in mind is,
that if we don't wrap j, R will return a vector. If we want a data.table
instead (most of the time this is what we want!) we need to wrap j,
either with list() or with .(). Generally, you will see .() more
commonly used as it is shorter.

\begin{Shaded}
\begin{Highlighting}[]
\CommentTok{\# returning a vector}
\NormalTok{student\_id\_vector }\OtherTok{\textless{}{-}}\NormalTok{ student\_data[, student\_id] }
\NormalTok{student\_id\_vector[}\DecValTok{1}\SpecialCharTok{:}\DecValTok{5}\NormalTok{]}
\end{Highlighting}
\end{Shaded}

\begin{verbatim}
## [1] 800251 800402 801902 803546 804776
## 849620 Levels: 00001 00002 00003 00004 00005 00006 00007 00008 00009 ... 9999
\end{verbatim}

\begin{Shaded}
\begin{Highlighting}[]
\CommentTok{\# returning a data.table, wrapping with list()}
\NormalTok{select\_columns }\OtherTok{\textless{}{-}}\NormalTok{ student\_data[, }\FunctionTok{list}\NormalTok{(student\_id, country, gender, read, science)]}
\NormalTok{select\_columns[}\DecValTok{1}\SpecialCharTok{:}\DecValTok{5}\NormalTok{]}
\end{Highlighting}
\end{Shaded}

\begin{verbatim}
##    student_id country gender    read science
## 1:     800251     ALB   male 375.984 445.039
## 2:     800402     ALB   male 434.352 421.731
## 3:     801902     ALB female 359.191 392.223
## 4:     803546     ALB   male 425.131 515.942
## 5:     804776     ALB   male 306.028 328.261
\end{verbatim}

\begin{Shaded}
\begin{Highlighting}[]
\CommentTok{\# returning a data.table, wrapping with .()}
\NormalTok{select\_columns }\OtherTok{\textless{}{-}}\NormalTok{ student\_data[,.(student\_id, country, gender, read, science)] }
\NormalTok{select\_columns[}\DecValTok{1}\SpecialCharTok{:}\DecValTok{5}\NormalTok{]}
\end{Highlighting}
\end{Shaded}

\begin{verbatim}
##    student_id country gender    read science
## 1:     800251     ALB   male 375.984 445.039
## 2:     800402     ALB   male 434.352 421.731
## 3:     801902     ALB female 359.191 392.223
## 4:     803546     ALB   male 425.131 515.942
## 5:     804776     ALB   male 306.028 328.261
\end{verbatim}

\begin{Shaded}
\begin{Highlighting}[]
\CommentTok{\# drop selected columns}
\NormalTok{select\_columns\_short }\OtherTok{\textless{}{-}}\NormalTok{ select\_columns[, }\SpecialCharTok{!}\FunctionTok{c}\NormalTok{(}\StringTok{"read"}\NormalTok{, }\StringTok{"science"}\NormalTok{)]}
\NormalTok{select\_columns\_short[}\DecValTok{1}\SpecialCharTok{:}\DecValTok{5}\NormalTok{]}
\end{Highlighting}
\end{Shaded}

\begin{verbatim}
##    student_id country gender
## 1:     800251     ALB   male
## 2:     800402     ALB   male
## 3:     801902     ALB female
## 4:     803546     ALB   male
## 5:     804776     ALB   male
\end{verbatim}

\hypertarget{renaming-columns}{%
\subsection{Renaming columns}\label{renaming-columns}}

\begin{Shaded}
\begin{Highlighting}[]
\CommentTok{\# selecting and renaming the columns for books and socioeconomic status}
\NormalTok{renamed\_cols }\OtherTok{\textless{}{-}}\NormalTok{ student\_data[, .(}\AttributeTok{books =}\NormalTok{ book, }\AttributeTok{socio\_eco\_status =}\NormalTok{ escs)]}
\FunctionTok{head}\NormalTok{(renamed\_cols)}
\end{Highlighting}
\end{Shaded}

\begin{verbatim}
##    books socio_eco_status
## 1:  0-10           0.6747
## 2: 11-25          -0.7566
## 3:  0-10          -2.5112
## 4:  0-10          -3.1843
## 5: 11-25          -1.7557
## 6: 11-25          -1.4855
\end{verbatim}

\hypertarget{creating-new-columns}{%
\subsection{Creating new columns}\label{creating-new-columns}}

{Tidyverse equivalent: mutate}

The := assignment symbol allows us to create new columns through the
modification/combination of one ore more columns. As part of the
assignment we can optionally specify the name of the new column. If we
don't specify a name, the default v1, v2\ldots{} will be used.

\begin{Shaded}
\begin{Highlighting}[]
\CommentTok{\# creating a new column which combines the reading and science scores}
\NormalTok{total\_score }\OtherTok{\textless{}{-}}\NormalTok{ male\_albanian\_students[, }
\NormalTok{  total\_score }\SpecialCharTok{:}\ErrorTok{=}\NormalTok{ read }\SpecialCharTok{+}\NormalTok{ science]}
\NormalTok{total\_score[}\DecValTok{1}\SpecialCharTok{:}\DecValTok{5}\NormalTok{]}
\end{Highlighting}
\end{Shaded}

\begin{verbatim}
##    year country school_id student_id mother_educ father_educ gender computer
## 1: 2018     ALB    800002     800251    ISCED 3A    ISCED 3A   male      yes
## 2: 2018     ALB    800002     800402     ISCED 2     ISCED 2   male      yes
## 3: 2018     ALB    800002     803546     ISCED 2     ISCED 2   male       no
## 4: 2018     ALB    800002     804776     ISCED 2    ISCED 3A   male      yes
## 5: 2018     ALB    800002     805287     ISCED 2     ISCED 2   male       no
##    internet    math    read science  stu_wgt desk room dishwasher television
## 1:      yes 490.187 375.984 445.039 13.51452  yes  yes       <NA>         3+
## 2:      yes 462.464 434.352 421.731 13.51452  yes  yes       <NA>          1
## 3:       no 482.501 425.131 515.942 13.51452  yes   no       <NA>          0
## 4:      yes 459.804 306.028 328.261 13.51452  yes  yes       <NA>          2
## 5:       no 441.037 271.213 391.562 13.51452   no  yes       <NA>          1
##    computer_n  car  book  wealth    escs total_score
## 1:          1    2  0-10 -0.0996  0.6747     821.023
## 2:          1    2 11-25 -0.7221 -0.7566     856.083
## 3:          0 <NA>  0-10 -7.0376 -3.1843     941.073
## 4:          1    0 11-25 -1.8375 -1.7557     634.289
## 5:          0    0  0-10 -3.3724 -3.2481     662.775
\end{verbatim}

\hypertarget{how-about-creating-multiple-columns}{%
\subsubsection{How about creating multiple
columns?}\label{how-about-creating-multiple-columns}}

\begin{Shaded}
\begin{Highlighting}[]
\CommentTok{\# creating multiple new columns}
\NormalTok{multiple\_columns }\OtherTok{\textless{}{-}}\NormalTok{ school\_data[ , }\StringTok{\textasciigrave{}}\AttributeTok{:=}\StringTok{\textasciigrave{}}\NormalTok{ (}
  \AttributeTok{total\_fun =}\NormalTok{ fund\_gov }\SpecialCharTok{+}\NormalTok{ fund\_fees }\SpecialCharTok{+}\NormalTok{ fund\_donation,  }\AttributeTok{total\_students =}\NormalTok{ enrol\_boys }\SpecialCharTok{+}\NormalTok{ enrol\_girls)]}
\NormalTok{multiple\_columns[}\DecValTok{1}\SpecialCharTok{:}\DecValTok{5}\NormalTok{]}
\end{Highlighting}
\end{Shaded}

\begin{verbatim}
##    year country school_id fund_gov fund_fees fund_donation enrol_boys
## 1: 2000     ALB     01001      100         0             0       1191
## 2: 2000     ALB     01004       98         1             1        334
## 3: 2000     ALB     01005       91         5             2        403
## 4: 2000     ALB     01010      100         0             0        114
## 5: 2000     ALB     01013        0        50            30        250
##    enrol_girls stratio public_private staff_shortage sch_wgt school_size
## 1:        1176   23.67         public           0.60       1        2367
## 2:         479   24.64         public          -0.95       1         813
## 3:         600      NA         public          -0.17       1        1003
## 4:         201   22.50         public           1.87       1         315
## 5:         248   26.92        private          -0.95       1         498
##    total_fun total_students
## 1:       100           2367
## 2:       100            813
## 3:        98           1003
## 4:       100            315
## 5:        80            498
\end{verbatim}

\hypertarget{descriptive-analysis}{%
\section{Descriptive analysis}\label{descriptive-analysis}}

\hypertarget{summarising-variables}{%
\subsection{Summarising variables}\label{summarising-variables}}

{Tidyverse equivalent: summarise}

First we drop NAs to make our lives easier.

\begin{Shaded}
\begin{Highlighting}[]
\NormalTok{clean\_school\_data }\OtherTok{\textless{}{-}} \FunctionTok{na.omit}\NormalTok{(school\_data)}
\end{Highlighting}
\end{Shaded}

We can perform any descriptive analysis on j, such as sum, maximum,
minimum. Optionally, we can assign a name to the outcome variable.

\begin{Shaded}
\begin{Highlighting}[]
\CommentTok{\# first we drop na{-}s to make our lives easier}
\NormalTok{clean\_school\_data }\OtherTok{\textless{}{-}} \FunctionTok{na.omit}\NormalTok{(school\_data)}

\CommentTok{\# school with most enrolled boys}
\NormalTok{max\_boys }\OtherTok{\textless{}{-}}\NormalTok{ clean\_school\_data[, }\FunctionTok{max}\NormalTok{(enrol\_boys)]}

\CommentTok{\# school with fewest enrolled girls }
\NormalTok{min\_girls }\OtherTok{\textless{}{-}}\NormalTok{ clean\_school\_data[, }\FunctionTok{min}\NormalTok{(enrol\_girls)]}

\CommentTok{\# adding logical operators: total enrolled girls exceeding enrolled boys}
\NormalTok{n\_school\_more\_girls }\OtherTok{\textless{}{-}}\NormalTok{ clean\_school\_data[, }\FunctionTok{sum}\NormalTok{(enrol\_boys }\SpecialCharTok{\textless{}}\NormalTok{ enrol\_girls)]}

\FunctionTok{print}\NormalTok{(}\FunctionTok{paste0}\NormalTok{(}\StringTok{"School with most enrolled boys: "}\NormalTok{, max\_boys))}
\end{Highlighting}
\end{Shaded}

\begin{verbatim}
## [1] "School with most enrolled boys: 8500"
\end{verbatim}

\begin{Shaded}
\begin{Highlighting}[]
\FunctionTok{print}\NormalTok{(}\FunctionTok{paste0}\NormalTok{(}\StringTok{"School with fewest enrolled girls "}\NormalTok{, min\_girls))}
\end{Highlighting}
\end{Shaded}

\begin{verbatim}
## [1] "School with fewest enrolled girls 0"
\end{verbatim}

\begin{Shaded}
\begin{Highlighting}[]
\FunctionTok{print}\NormalTok{(}\FunctionTok{paste0}\NormalTok{(}\StringTok{"Number of schools with more enrolled girls than boys "}\NormalTok{, n\_school\_more\_girls))}
\end{Highlighting}
\end{Shaded}

\begin{verbatim}
## [1] "Number of schools with more enrolled girls than boys 33517"
\end{verbatim}

\begin{Shaded}
\begin{Highlighting}[]
\CommentTok{\# sub{-}setting prior to a computation}
\NormalTok{subseting\_computing }\OtherTok{\textless{}{-}}\NormalTok{ clean\_school\_data[country}\SpecialCharTok{==} \StringTok{"FRA"} \SpecialCharTok{\&}\NormalTok{ year }\SpecialCharTok{==} \DecValTok{2018}\NormalTok{, }\FunctionTok{sum}\NormalTok{(total\_students)]}
\end{Highlighting}
\end{Shaded}

\hypertarget{special-symbol-.n}{%
\subsection{Special symbol .N}\label{special-symbol-.n}}

.N returns the number of observations in the current group. .N is
particularly powerful when combined with grouping in ``by''.

\begin{Shaded}
\begin{Highlighting}[]
\CommentTok{\# we can use length() to get the number of observations}
\NormalTok{clean\_school\_data[country}\SpecialCharTok{==} \StringTok{"FRA"} \SpecialCharTok{\&}\NormalTok{ year }\SpecialCharTok{==} \DecValTok{2018}\NormalTok{, }\FunctionTok{length}\NormalTok{(school\_size)] }
\end{Highlighting}
\end{Shaded}

\begin{verbatim}
## [1] 174
\end{verbatim}

\begin{Shaded}
\begin{Highlighting}[]
\CommentTok{\# but we can also use the special operator .N}
\NormalTok{clean\_school\_data[country}\SpecialCharTok{==} \StringTok{"FRA"} \SpecialCharTok{\&}\NormalTok{ year }\SpecialCharTok{==} \DecValTok{2018}\NormalTok{, .N]}
\end{Highlighting}
\end{Shaded}

\begin{verbatim}
## [1] 174
\end{verbatim}

\begin{Shaded}
\begin{Highlighting}[]
\CommentTok{\# we see that we get the same result, but with shorter code}
\end{Highlighting}
\end{Shaded}

\hypertarget{grouping-with-by}{%
\subsection{Grouping with ``by''}\label{grouping-with-by}}

\emph{By} allows us to perform operations by group. We are not limited
to a single group, but can specify several groups:

\begin{Shaded}
\begin{Highlighting}[]
\CommentTok{\# getting the total number of schools by country}
\NormalTok{group\_country }\OtherTok{\textless{}{-}}\NormalTok{ school\_data[, .(.N), by }\OtherTok{=}\NormalTok{ .(country)]}
\NormalTok{group\_country[}\DecValTok{1}\SpecialCharTok{:}\DecValTok{5}\NormalTok{]}
\end{Highlighting}
\end{Shaded}

\begin{verbatim}
##    country    N
## 1:     ALB 1116
## 2:     ARG 1212
## 3:     AUS 3557
## 4:     AUT 1638
## 5:     BEL 1903
\end{verbatim}

\textbf{Conditional grouping:} ``\emph{By} also accepts conditional
expressions, i.e., we can create groups around conditions. The outcome
is a matrix which shows all possible true/false combinations.

Below we compute the mean staff shortage for Albanian schools
differentiated by (i) whether more than 50\% of a school's funding comes
from donations, (ii) whether a school's total number of students is
below 500

\begin{Shaded}
\begin{Highlighting}[]
\NormalTok{by\_condition }\OtherTok{\textless{}{-}}\NormalTok{ clean\_school\_data[country }\SpecialCharTok{==} \StringTok{"ALB"}\NormalTok{,}
\NormalTok{        .(}\FunctionTok{mean}\NormalTok{(staff\_shortage)),}
\NormalTok{        by }\OtherTok{=}\NormalTok{ .(fund\_donation }\SpecialCharTok{\textgreater{}} \DecValTok{50}\NormalTok{, total\_students }\SpecialCharTok{\textless{}} \DecValTok{500}\NormalTok{)]}
\NormalTok{by\_condition[}\DecValTok{1}\SpecialCharTok{:}\DecValTok{5}\NormalTok{]}
\end{Highlighting}
\end{Shaded}

\begin{verbatim}
##    fund_donation total_students          V1
## 1:         FALSE          FALSE -0.09020168
## 2:         FALSE           TRUE -0.20778860
## 3:          TRUE           TRUE -0.39405000
## 4:          TRUE          FALSE -0.15485000
## 5:            NA             NA          NA
\end{verbatim}

\hypertarget{combining-several-operations}{%
\subsection{Combining several
operations}\label{combining-several-operations}}

\begin{Shaded}
\begin{Highlighting}[]
\CommentTok{\# let\textquotesingle{}s try to subset both in columns and rows:}
\NormalTok{cols\_rows }\OtherTok{\textless{}{-}}\NormalTok{ clean\_school\_data[country }\SpecialCharTok{==} \StringTok{"FRA"} \SpecialCharTok{\&}\NormalTok{ year }\SpecialCharTok{==} \DecValTok{2018}\NormalTok{,}
\NormalTok{               .(}\AttributeTok{m\_total =} \FunctionTok{mean}\NormalTok{(total\_students), }\AttributeTok{m\_fun =} \FunctionTok{mean}\NormalTok{(total\_fun))]}
\NormalTok{cols\_rows[}\DecValTok{1}\SpecialCharTok{:}\DecValTok{5}\NormalTok{]}
\end{Highlighting}
\end{Shaded}

\begin{verbatim}
##    m_total    m_fun
## 1: 888.569 96.81609
## 2:      NA       NA
## 3:      NA       NA
## 4:      NA       NA
## 5:      NA       NA
\end{verbatim}

\begin{Shaded}
\begin{Highlighting}[]
\CommentTok{\# mean number of boys and girls in schools in KAZ, grouped by year and private/public status }
\NormalTok{all\_combined }\OtherTok{\textless{}{-}}\NormalTok{ clean\_school\_data[country }\SpecialCharTok{==} \StringTok{"KAZ"}\NormalTok{,}
\NormalTok{        .(}\FunctionTok{mean}\NormalTok{(enrol\_boys), }\FunctionTok{mean}\NormalTok{(enrol\_girls)),}
\NormalTok{        by }\OtherTok{=}\NormalTok{ .(year, public\_private)]}
\NormalTok{all\_combined[}\DecValTok{1}\SpecialCharTok{:}\DecValTok{5}\NormalTok{]}
\end{Highlighting}
\end{Shaded}

\begin{verbatim}
##    year public_private       V1       V2
## 1: 2009         public 356.1211 357.3263
## 2: 2009        private 236.3750 355.5000
## 3: 2012         public 367.0148 352.9951
## 4: 2012        private 430.4286 493.4286
## 5: 2018         public 356.6547 332.8354
\end{verbatim}

\begin{Shaded}
\begin{Highlighting}[]
\CommentTok{\# why are we seeing V1, V2?}
\end{Highlighting}
\end{Shaded}

\hypertarget{advanced-applications}{%
\section{Advanced applications}\label{advanced-applications}}

\hypertarget{sd-and-.sdcols}{%
\subsection{.SD and .SDcols}\label{sd-and-.sdcols}}

What if we need to repeat an operation across many columns, like finding
the mean? Is there a way to calculate the mean for several columns at
once? Yes, we can subset each column with .SD, and then compute the mean
with with lapply.

.SD is particularly useful in the context of grouping, where it refers
to each of these sub-data.tables, one-at-a-time. If we want to limit the
column-wise computation to selected columns only (e.g., only numerical
ones), we specify these through .SDcols. .SDcols also accepts a vector
as input, which can be defined outside of data.table.

\begin{Shaded}
\begin{Highlighting}[]
\CommentTok{\# mean of 3 selected columns across countries in 2018}
\NormalTok{mean\_columns }\OtherTok{\textless{}{-}} \FunctionTok{c}\NormalTok{(}\StringTok{"enrol\_boys"}\NormalTok{, }\StringTok{"enrol\_girls"}\NormalTok{, }\StringTok{"fund\_fees"}\NormalTok{) }\CommentTok{\# specifying the columns to iterate over}
\CommentTok{\# number of enrolled boys, enrolled girls and hare of funding from student fees (full percentage points)}
\NormalTok{selected\_means }\OtherTok{\textless{}{-}}\NormalTok{ clean\_school\_data[year }\SpecialCharTok{==} \DecValTok{2018}\NormalTok{,        }
        \FunctionTok{lapply}\NormalTok{(.SD, mean),                    }
\NormalTok{        by }\OtherTok{=}\NormalTok{ .(country),           }
\NormalTok{        .SDcols }\OtherTok{=}\NormalTok{ mean\_columns]   }
\NormalTok{selected\_means[}\DecValTok{1}\SpecialCharTok{:}\DecValTok{5}\NormalTok{]}
\end{Highlighting}
\end{Shaded}

\begin{verbatim}
##    country enrol_boys enrol_girls fund_fees
## 1:     ALB   231.8041    200.8763 16.659794
## 2:     QAZ   719.4483    610.8448  1.603448
## 3:     ARG   264.7633    256.6078 27.674912
## 4:     AUS   515.7223    477.7769 21.745455
## 5:     BIH   197.5504    177.3876  3.992248
\end{verbatim}

\hypertarget{chaining}{%
\subsection{Chaining}\label{chaining}}

{Tidyverse equivalent: piping}

To avoid saving intermediate results, data.table uses chaining. The
syntax can be read as `result from first {[}{]}-operation is used in
subsequent {[}{]}-operation'. Chaining vertically is best practice and
enhances readability.

DT{[}\ldots{} {]}{[}\ldots{} {]}{[}\ldots{} {]}

\begin{Shaded}
\begin{Highlighting}[]
\CommentTok{\# computing the mean staff shortage by country and then ordering the result in decreasing order}
\NormalTok{chaining }\OtherTok{\textless{}{-}}\NormalTok{ clean\_school\_data[,}
\NormalTok{                              .(}\AttributeTok{staff\_shortage =} \FunctionTok{mean}\NormalTok{(staff\_shortage)), }
\NormalTok{                              by }\OtherTok{=}\NormalTok{ .(country)}
\NormalTok{                  ][}\FunctionTok{order}\NormalTok{(}\SpecialCharTok{{-}}\NormalTok{staff\_shortage)}
\NormalTok{                    ]}
\NormalTok{chaining[}\DecValTok{1}\SpecialCharTok{:}\DecValTok{5}\NormalTok{]}
\end{Highlighting}
\end{Shaded}

\begin{verbatim}
##    country staff_shortage
## 1:     KGZ      1.2432147
## 2:     TUR      1.0665621
## 3:     MAR      1.0262737
## 4:     JOR      0.9953706
## 5:     QCH      0.7722232
\end{verbatim}

\end{document}
